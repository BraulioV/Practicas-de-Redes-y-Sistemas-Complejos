\documentclass[paper=a4, fontsize=11pt]{article} % A4 paper and 11pt font size
\usepackage[a4paper, margin=1.3in]{geometry}
% ---- Entrada y salida de texto -----

\usepackage[T1]{fontenc} % Use 8-bit encoding that has 256 glyphs
\usepackage[utf8]{inputenc}
% \usepackage[light,math]{iwona}

\usepackage{fancyhdr}
\usepackage{fancybox}
\usepackage{pseudocode}


% ---- Idioma --------

\usepackage[spanish, es-tabla]{babel} % Selecciona el español para palabras introducidas automáticamente, p.ej. "septiembre" en la fecha y especifica que se use la palabra Tabla en vez de Cuadro

% ---- Otros paquetes ----

\usepackage{amsmath,amsfonts,amsthm} % Math packages
\usepackage{graphics,graphicx, floatrow} %para incluir imágenes y notas en las imágenes
\usepackage{graphics,graphicx, float} %para incluir imágenes y colocarlas
\usepackage{enumerate}
\usepackage{subfigure}
% \makesavenoteenv{tabular}
% \makesavenoteenv{table}
% Para hacer tablas comlejas
%\usepackage{multirow}
%\usepackage{threeparttable}
\usepackage{sectsty} % Allows customizing section commands
\allsectionsfont{\centering \scshape} % Make all sections centered, the default font and small caps

\usepackage{fancyhdr} % Custom headers and footers
\usepackage[usenames, dvipsnames]{color}
\usepackage{colortbl}
% \usepackage{minted}
\usepackage{xcolor}
\usepackage{url}
\usepackage{cancel}

% \newmintedfile[mycpp]{c++}{
%     linenos,
%     numbersep=5pt,
%     gobble=0,
%     frame=lines,
%     framesep=2mm,
% }

% \newmintedfile[myc]{c}{
%     linenos,
%     numbersep=5pt,
%     gobble=0,
%     frame=lines,
%     framesep=2mm,
% }

% \newmintedfile[mypython]{python}{
%     linenos,
%     numbersep=5pt,
%     gobble=0,
%     frame=lines,
%     framesep=2mm,
% }

\usepackage{cite}

\usepackage[bookmarks=true,
    bookmarksnumbered=false, % true means bookmarks in
             % left window are numbered
    bookmarksopen=false,   % true means only level 1
             % are displayed.
    colorlinks=true,
    urlcolor=webblue,
    citecolor=webred,
    linkcolor=webblue]{hyperref}
\definecolor{webgreen}{rgb}{0, 0.5, 0} % less intense green
\definecolor{webblue}{rgb}{0, 0, 0.5}  % less intense blue
\definecolor{webred}{rgb}{0.5, 0, 0} % less intense red

%% Define a new 'leo' style for the package that will use a smaller font.
\makeatletter
\def\url@leostyle{%
  \@ifundefined{selectfont}{\def\UrlFont{\sf}}{\def\UrlFont{\small\ttfamily}}}
\makeatother
%% Now actually use the newly defined style.
\urlstyle{leo}

\pagestyle{fancyplain} % Makes all pages in the document conform to the custom headers and footers
\fancyhead{} % No page header - if you want one, create it in the same way as the footers below
\fancyfoot[L]{} % Empty left footer
\fancyfoot[C]{} % Empty center footer
\fancyfoot[R]{\thepage} % Page numbering for right footer
\renewcommand{\headrulewidth}{0pt} % Remove header underlines
\renewcommand{\footrulewidth}{0pt} % Remove footer underlines
\setlength{\headheight}{13.6pt} % Customize the height of the header

\numberwithin{equation}{section} % Number equations within sections (i.e. 1.1, 1.2, 2.1, 2.2 instead of 1, 2, 3, 4)
\numberwithin{figure}{section} % Number figures within sections (i.e. 1.1, 1.2, 2.1, 2.2 instead of 1, 2, 3, 4)
\numberwithin{table}{section} % Number tables within sections (i.e. 1.1, 1.2, 2.1, 2.2 instead of 1, 2, 3, 4)

\setlength\parindent{0pt} % Removes all indentation from paragraphs - comment this line for an assignment with lots of text

\newcommand{\horrule}[1]{\rule{\linewidth}{#1}} % Create horizontal rule command with 1 argument of height

%%%%% Para cambiar el tipo de letra en el título de la sección %%%%%%%%%%%
% \usepackage{sectsty}
% \chapterfont{\fontfamily{pag}\selectfont} %% for chapter if you want
% \sectionfont{\fontfamily{pag}\selectfont}
% \subsectionfont{\fontfamily{pag}\selectfont}
% \subsubsectionfont{\fontfamily{pag}\selectfont}

%----------------------------------------------------------------------------------------
% TÍTULO Y DATOS DEL ALUMNO
%----------------------------------------------------------------------------------------

\title{ 
\normalfont \normalsize 
\textsc{{\bf Redes y Sistemas Complejos (2016-2017)} \\ Grado en Ingeniería Informática \\ Universidad de Granada} \\ [25pt] % Your university, school and/or department name(s)
\horrule{0.5pt} \\[0.4cm] % Thin top horizontal rule
\huge Memoria Práctica 3\\ Estudio Comparativo de Métodos para Poda y Visualización de Redes.\\% The assignment title
\horrule{2pt} \\[0.5cm] % Thick bottom horizontal rule
}

\author{Braulio Vargas López\\DNI: 20079894C\\Correo: brauliovarlop@correo.ugr.es} % Nombre y apellidos

\date{\normalsize\today} % Incluye la fecha actual

%----------------------------------------------------------------------------------------
% DOCUMENTO
%----------------------------------------------------------------------------------------

\begin{document}

\maketitle % Muestra el Título
\pagenumbering{gobble}
\newpage %inserta un salto de página

\tableofcontents % para generar el índice de contenidos
% \newpage
\pagenumbering{arabic}

\section{Poda y Visualización}

En esta primera parte del guión, vamos a realizar un análisis de diez de las veinte redes disponibles, podándo cada una de las redes con la versión \textit{BinaryPathfinder} del algoritmo. Las diez redes escogidas son:

\begin{enumerate}
    \item France-2002
    \item Germany-2002
    \item Japan-2002
    \item Spain-1996
    \item Spain-1998
    \item Spain-2002
    \item Spain-2004
    \item United\_Kingdom-2002
    \item United\_States-2002
    \item World
\end{enumerate}

La razón de coger estas redes es por el hecho de poder ver la evolución de la producción científica en España a lo largo de ocho años, y poder compararlas con otros países durante el año 2002. Además, de poder ver la producción científica a nivel mundial en la última red.

A continuación, podemos ver los resultados obtenidos aplicando el algoritmo a cada uno de las redes seleccionadas:

%%%%%%%%%%%%%%%%%%%%%%%%%%%%%%%%%%%%%%%%%%%%%%%%%%%%%%%%%%%%%%
\begin{table}[H]
  \begin{minipage}{0.45\textwidth}
      \begin{tabular}{ccc}
       France-2002 & & \\
          \hline
            n=267             & $L$ &  $D$  \\
            \hline
            Red original &               23986 & 0.675453   \\
            2            &                 312 & 0.00878601 \\
            3            &                 275 & 0.00774408 \\
            4            &                 271 & 0.00763144 \\
            5            &                 270 & 0.00760328 \\
            266          &                 268 & 0.00754696 \\
            \hline
     \end{tabular}
  \end{minipage}
  \begin{minipage}{0.45\textwidth}
      \begin{tabular}{ccc}
         Germany-2002 & & \\
         \hline
         n=269              &   $L$ &   $D$ \\
         \hline
         Red original &               25395 & 0.704516   \\
         2            &                 313 & 0.00868335 \\
         3            &                 277 & 0.00768463 \\
         4            &                 272 & 0.00754591 \\
         5            &                 270 & 0.00749043 \\
         268          &                 269 & 0.00746269 \\
        \hline
     \end{tabular}
  \end{minipage}
  \label{fr-gr}
  \caption{Tablas para las redes de France-2002.net y Germany-2002.net}
\end{table}
%%%%%%%%%%%%%%%%%%%%%%%%%%%%%%%%%%%%%%%%%%%%%%%%%%%%%%%%%%%%%%%
\begin{table}[H]
  \begin{minipage}{0.45\textwidth}
    \begin{tabular}{ccc}
      Japan-2002 & & \\
      \hline
      n=265              &   $L$ &   $D$ \\
      \hline
      Red original &               21754 & 0.621898   \\
      2            &                 316 & 0.00903373 \\
      3            &                 279 & 0.00797599 \\
      4            &                 269 & 0.00769011 \\
      5            &                 267 & 0.00763293 \\
      264          &                 267 & 0.00763293 \\
      \hline
    \end{tabular}
  \end{minipage}
  \begin{minipage}{0.45\textwidth}
    \begin{tabular}{ccc}
      Spain-1996 & & \\
      \hline
      n=243              &   $L$ &   $D$ \\
      \hline
      Red original &                5967 &  0.202938  \\
      2            &                 394 &  0.0134    \\
      3            &                 313 &  0.0106452 \\
      4            &                 303 &  0.0103051 \\
      5            &                 300 &  0.010203  \\
      242          &                 300 &  0.010203  \\
      \hline
    \end{tabular}
  \end{minipage}
  \label{jp-sp}
  \caption{Tablas para las redes Japan-2002.net y Spain-1996.net}
\end{table}
%%%%%%%%%%%%%%%%%%%%%%%%%%%%%%%%%%%%%%%%%%%%%%%%%%%%%%%%%%%%%%%
\begin{table}[H]
\begin{minipage}{0.45\textwidth}
  \begin{tabular}{ccc}
  Spain-1998 & & \\
  \hline
  n=258              &   $L$ &   $D$ \\
  \hline
  Red original &               12971 & 0.391247   \\
  2            &                 320 & 0.00965222 \\
  3            &                 279 & 0.00841553 \\
  4            &                 267 & 0.00805357 \\
  5            &                 267 & 0.00805357 \\
  257          &                 267 & 0.00805357 \\
  \hline
  \end{tabular}
\end{minipage}
%%%%%%%%%%%%%%%%%%%%%%%%%%%%%%%%%%%%%%%%%%%%%%%%%%%%%%%%%%%%%%%
\begin{minipage}{0.45\textwidth}
  \begin{tabular}{ccc}
    Spain-2002 & & \\
    \hline
    n=264              &   $L$ &   $D$ \\
    \hline
    Red original &               21807 & 0.628154   \\
    2            &                 320 & 0.00921765 \\
    3            &                 274 & 0.00789261 \\
    4            &                 265 & 0.00763337 \\
    5            &                 263 & 0.00757576 \\
    263          &                 263 & 0.00757576 \\
    \hline
  \end{tabular}
\end{minipage}
\label{sp98-sp02}
\caption{Tablas para las redes \textit{Spain-1998.net} y \textit{Spain-2002.net}.}
\end{table}
%%%%%%%%%%%%%%%%%%%%%%%%%%%%%%%%%%%%%%%%%%%%%%%%%%%%%%%%%%%%%%%
\begin{table}[H]
\begin{minipage}{0.45\textwidth}
  \begin{tabular}{ccc}
    Spain-2004 & & \\
    \hline
    (n=269)              &   $L$ &   $D$ \\
    \hline
    Red original &               24991 & 0.693309   \\
    2            &                 332 & 0.00921045 \\
    3            &                 280 & 0.00776785 \\
    4            &                 272 & 0.00754591 \\
    5            &                 271 & 0.00751817 \\
    268          &                 270 & 0.00749043 \\
    \hline
  \end{tabular}
\end{minipage}
\begin{minipage}{0.45\textwidth}
  \begin{tabular}{ccc}
    United\_Kingdom-2002 & & \\
    \hline
    (n=276)              &   $L$ &   $D$ \\
    \hline
    Red original &               28707 & 0.756443   \\
    2            &                 326 & 0.00859025 \\
    3            &                 288 & 0.00758893 \\
    4            &                 280 & 0.00737813 \\
    5            &                 279 & 0.00735178 \\
    275          &                 276 & 0.00727273 \\
    \hline
  \end{tabular}
\end{minipage}
\label{sp04-uk02}
\caption{Tablas para las redes \textit{Spain-2004.net} y \textit{United\_Kingdom-2002.net}.}
\end{table}
%%%%%%%%%%%%%%%%%%%%%%%%%%%%%%%%%%%%%%%%%%%%%%%%%%%%%%%%%%%%%%%
\begin{table}[H]
\begin{minipage}{0.5\textwidth}
  \begin{tabular}{ccc}
    United\_States-2002 & & \\
    \hline
    (n=276)              &   $L$ &   $D$ \\
    \hline
    Red original &               31292 & 0.824559   \\
    2            &                 314 & 0.00827404 \\
    3            &                 287 & 0.00756258 \\
    4            &                 279 & 0.00735178 \\
    5            &                 277 & 0.00729908 \\
    275          &                 275 & 0.00724638 \\
    \hline
  \end{tabular}  
\end{minipage}
\begin{minipage}{0.35\textwidth}
  \begin{tabular}{ccc}
    World.net & & \\
    \hline
    (n=218)              &   $L$ &   $D$ \\
    \hline
    Red original &               20154 & 0.85207    \\
    2            &                 280 & 0.0118378  \\
    3            &                 233 & 0.00985076 \\
    4            &                 223 & 0.00942798 \\
    5            &                 220 & 0.00930115 \\
    217          &                 217 & 0.00917431 \\
    \hline
  \end{tabular}
\end{minipage}
\label{usw}
\caption{Tabla para las redes \textit{United\_States-2002.net} y \textit{World.net}.}
\end{table}
  

Como podemos ver en las tablas anteriores, los resultados para todas las redes son parecidos. Con $n=2$, el algoritmo poda una gran cantidad de enlaces en todas las redes, lo que hace bajar muchísimo la densidad de la red. Por ejemplo, en la \hyperref[fr-gr]{Tabla \ref*{fr-gr}}, para la red \textit{France-2002.net} podemos ver cómo la red original tenía 23986 enlaces y una densidad $D\approx 0.7$, pasa a tener con 312 enlaces y una densidad de $D \approx 0.009$, unas 77 veces menor aproximadamente.

Además de esto, en todas las redes se da el mismo suceso y es que el número de enlaces para $n=5$ y $n = \infty$ es prácticamente igual, o hay casos en los que es igual, como se puede ver en las tablas \hyperref[jp-sp]{Tabla \ref*{jp-sp}}. Esto se debe a que el algoritmo, a partir de este punto, no puede podar más.

\subsection{Visualizaciones de las Redes}

\section{Análisis de eficiencia de las variantes del algoritmo \textit{Pathfinder}}

\end{document}
